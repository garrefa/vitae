%!TEX TS-program = xelatex
\documentclass[]{friggeri-cv}
\addbibresource{bibliography.bib}

\begin{document}
\header{Alexandre}{ Garrefa}
       {software developer}

% In the aside, each new line forces a line break
\begin{aside}
  \section{about}
    Sao Paulo
    Brazil
    ~
    \href{mailto:alexmrg@gmail.com}{alexmrg@gmail.com}
    \href{https://linkedin.com/in/garrefa/}{linkedin.com/in/garrefa/}
    \href{https://github.com/garrefa}{github.com/garrefa}
    \href{https://twitter.com/alexmrg}{twitter.com/alexmrg}
  \section{languages}
    portuguese native
    english fluent
    german notions
  \section{programming}
    Swift, ObjC
    JavaScript
    (ES5/6, node.js)
    ~
    basic knowledge:
    Ruby, Python, Shell Script, C
\end{aside}

\section{interests}

iOS Applications, backend, networking, application security, agile and XP, API design, design patterns, testable software, continuous deployment, continuous integration, modern languages, linux/unix systems, OSS, knowledge sharing, user experience, user interface

\section{experience}

\begin{entrylist}
  \entry
    {Nov 15/Now}
    {Singu (www.singu.com.br)}
    {CTO and iOS Developer}
    {\emph{Singu wants to be the Uber of beauty. Responsible to develop the iOS app and make sure everything else works. Review backend and payment logic. Take care of AWS infrastructure, supervision Android development. Managing a team of four other developers.}}
  \entry
    {Nov 14/Nov 15}
    {Concrete Solutions}
    {iOS Tech Leader}
    {\emph{Concrete is a big consultancy always full of projects. Banks, TV stations, e-commerces, and others. Designed and build the network and security layers of a bank framework which is used in all of they're iOS apps. Was responsible to help the other iOS programmers, evaluate new projects, interview candidates and facilitate the knowledge sharing so that everybody could learn and become a better programmer each day.}}
  \entry
    {May 13/Nov 14}
    {MakeYou Inc, (www.make.me)}
    {Senior iOS Developer}
    {\emph{Main iOS developer, responsible for building, testing and deploying the app. Working remotely with a distributed team.
MakeMe is a group challenge platform. It supports mobile group challenges to help coaches and trainers create and host challenge plans that engage and impact clients anywhere in the world. It  helps promoting: health and wellness; personal or professional, learning and development goals; using mobile group challenges.}}
  \entry
    {Jan 13/Apr 13}
    {OnPlaces (www.onplaces.com)}
    {iOS Developer}
    {\emph{Developed the first version of they're social network app. With the app people can check the most interesting places to hangout and their Facebook friends in each of those places. Once there, they can check in and see who else is in the same place, send messages, post texts and comments about the place.}}
  \entry
    {08/Jan 13}
    {Heurys Tecnologia LTDA (www.heurys.com.br)}
    {Mobile Office Director, iOS Developer}
    {\emph{Responsible for mobile projects, developing them when performed in iOS native platform and performing management when developed with GeneXus.}}
  \entry
    {06/Dec 07}
    {Agathos Tecnologia (www.agathos.com.br)}
    {Team Leader for voice infrastructure solutions}
    {\emph{Coordinator of the technical team of voice infrastructure solutions. Working with voice centers from Dígitro and Ericsson.}}
  \entry
    {03/Jun 06}
    {SPD Tecnologia (www.spd.com.br)}
    {From Internship to Network Security Analyst}
    {\emph{Network security analyst, expert in Linux operating systems. Configuration of firewall, proxy, DNS, load balancing, among others.}}
\end{entrylist}

\newpage

\section{applications}

\begin{entrylist}

      \entry
    {2015-Now}
    {Singu (www.singu.com.br)}
    {Customers and Professionals Apps}
    {Responsible to work with the design team to prototype the apps and to develop it from start to finish. Implement push notifications, centralized navigation, screens layout, animations, network communication, error handling, analytics, crash report, tests, continuous delivery system.}
    
    \entry
    {2014-2015}
    {Concrete Solutions (www.concretesolutions.com.br)}
    {Many Apps}
    {Itaú Bank network and security framework, Redecard payment app, super visioned some e-commerce projects like Dafiti, Netshoes, Centauro. Some Globo.com apps and frameworks. And others.}

      \entry
    {2012-2014}
    {Meias Lupo (http://site.lupo.com.br/)}
    {Enterprise: Sales Force System App + Electronic Catalog}
    {The project underwent production with the implementation of the SAP software in February 2013 and has been in operation ever since. Responsible for conducting survey of requirements with the customer, UX definitions, backend specifications for integration with the SAP software, the whole development of the iPad app, as well as being the link between the backend team and the SAP team. Participation in meetings with users, providing training and support during the operation. At the the time, the project was being used throughout Lupo's sales department, with around 130 representatives spread across the country, and, by January 2014, orders placed on the SAP reached more than 40,000.}

    \entry
    {2013}
    {Restaurante Web App (http://www.restauranteweb.com.br/)}
    {Food Delivery App by Just Eat}
    {Consultancy and coach of the Dart Digital development team to adapt the app for Brazil.}
    
    \entry
    {2012}
    {Meu Saara App - Apps Criativos (www.appscriativos.com)}
    {}
    {Consultancy and development of the app. It is a guide to Saara, a shopping center in the city of Rio de Janeiro. The user can browse the stores by categories, check its location on the map, call the store, get directions to it and bookmark favorite stores.}
    
    \entry
    {2012}
    {Pic2Party App - Apps Criativos (www.appscriativos.com)}
    {}
    {App focused on sharing event photos. The host of the event can invite people to a shared photo album. Once invited to the group, all members have access to all photos. Development of the initial version of the app, iOS based UX consulting, backup system improvement and reduction of the size of the app.}

    \entry
    {2012}
    {Mercado Ribeirão App}
    {}
    {Consultancy focused on improving the performance of data loading and publishing the app in the App Store. Analysis of the loading routine code. By the end of the work, data loading time was reduced from 3 minutes to 10 seconds.}
    
    \entry
    {2012}
    {TipsTap App}
    {}
    {Consultancy on code quality and push notification deployment.}
    
\end{entrylist}

\section{education}

\begin{entrylist}
  \entry
    {2007}
    {BS. {\normalfont Computer Science}}
    {UNISEB Centro Universitário, Sao Paulo}
    {}
\end{entrylist}

\end{document}
